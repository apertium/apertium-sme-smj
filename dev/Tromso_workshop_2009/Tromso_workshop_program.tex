\documentclass[a4paper,english,12pt]{article}
\usepackage{babel,norsk}
\usepackage{ucs} %sami letters
% \usepackage{amssymb} %mathematical
\usepackage[utf8x]{inputenc}
\usepackage[T1]{fontenc}
\usepackage{harvard}

\usepackage[dvips]{graphicx}
\usepackage{rotating} 

%\usepackage{tikz}
%\usepackage{array}
%\usepackage{arydshln} %has to be after array
%\usepackage{multirow}
\usepackage{graphics}
\usepackage{graphicx}
\usepackage{tabularx} %specified width
\usepackage{tipa}
\usepackage{booktabs}
\usepackage{ctable} %loads booktable by default
\usepackage{colortbl}
\usepackage{covington}
\usepackage{url}
\usepackage{harvard}
\usepackage[right=2.5cm,left=2.5cm,top=2cm,bottom=2cm]{geometry}
% \usepackage{bibtexlogo}
\usepackage{setspace}

\usepackage{fancyhdr}
\usepackage{linguex}

\begin{document}

\setcounter{secnumdepth}{3}
\setcounter{tocdepth}{3}
%\linespread{1.5}
\begin{spacing}{0.8}


\newcommand{\tx}{\mbox{t\hspace{-.35em}-}} % for S‡m




\title{Workshop i Nordsamisk-lulesamisk maskinomsetjing 29.-30.09.2009}

\maketitle

\section{Innføring i maskinomsetjing - (29.09.2009)}


\subsection{SVHUM E0102: 9-12}
\begin{enumerate}
\item Opning, presentasjon 
\item Google omsetjing vs. Apertium omsetjing
\item Installere Apertium  %Apertium setup
\end{enumerate}

\subsection{Teorifagsbygget 4553: 13-16}

\begin{enumerate}
\item Korleis fungerer maskinomsetjing?  %Apertium system
\item Kva er god maskinomsetjing? 
\item Nordsamisk - lulesamisk kontrastiv grammatikk 
\end{enumerate}


\section{Praktisk arbeid med maskinomsetjing  (30.09.2009)}

\subsection{SVHUM E0102: 9-12}

\begin{enumerate}
\item Ávvir på Lulesamisk - finne feil 
\item Korleis rapportere feil? 
\item Det tospråklige leksikonet
\end{enumerate}

\subsection{SVHUM E0102: 13-16}

\begin{enumerate}
\item Syntaktisk transfer 
\item Grammatiske spørsmål 
\item Diskusjon, oppsummering, planar framover
\end{enumerate}


\end{document}