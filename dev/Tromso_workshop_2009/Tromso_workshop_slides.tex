\documentclass{beamer}
\setbeamertemplate{navigation symbols}{}

\usepackage[english]{babel}
\usepackage{ucs} %sami letters
% \usepackage{amssymb} %mathematical
\usepackage[utf8x]{inputenc}
\usepackage[T1]{fontenc}
\usepackage{harvard}
%\usepackage{rotating} 

%\usepackage{tikz}
%\usepackage{array}
%\usepackage{arydshln} %has to be after array
%\usepackage{multirow}
\usepackage{graphics}
\usepackage{tabularx} %specified width
\usepackage{tipa}
\usepackage{booktabs}
\usepackage{ctable} %loads booktable by default
\usepackage{colortbl}
%\usepackage{covington}
\usepackage{url}
%\usepackage[right=2.5cm,left=2.5cm,top=2cm,bottom=2cm]{geometry}
% \usepackage{bibtexlogo}
\usepackage{setspace}

\usepackage{fancyhdr}
\usepackage{linguex}


\usetheme{Montpellier} %Montpellier

\beamersetuncovermixins{\opaqueness<1>{25}}{\opaqueness<2->{15}}
\begin{document}
\title{Workshop i Nordsamisk-lulesamisk maskinomsetjing}  
\author{Linda Wiechetek, Francis Tyers, Trond Trosterud}
\date{\today} 
\begin{frame}
\titlepage
\end{frame}

\begin{frame}\frametitle{Innhold}\tableofcontents
\end{frame} 


\section{Bures boahtin!} 
\begin{frame}\frametitle{Bures boahtin!} 
\begin{itemize}
\item maskinoversetting i bilingual publisering (daglige aviser)
\item daglige aviser
\item skolebøker
\item flere tekster betyr også større etterspørsel
\end{itemize}
\end{frame}


%\subsection{Subsection no.1.1  }
\begin{frame}\frametitle{Forventninger}  
\begin{itemize}
\item Hva forventer du av en oversettelse? \pause
\item Hva forventer du av maskinoversettelse?
\end{itemize}

\end{frame}

\begin{frame}\frametitle{Maskinoversetting}  
\begin{columns}
\begin{column}{5cm}
\begin{itemize}
\item<1-> Google
\item<3-> Babelfish
\item<5-> PROMT
\item<7-> Gramtrans
\end{itemize}
\vspace{3cm} 
\end{column}
\begin{column}{5cm}
\begin{overprint}
\scalebox{.2}[.2]{\includegraphics<2>{google.png}}
%\includegraphics<4>{PIC2}
%\includegraphics<6>{PIC3}
\scalebox{.2}[.2]{\includegraphics<6>{gramtrans.png}}
\end{overprint}
\end{column}
\end{columns}
\end{frame}


\begin{frame} 

\begin{exampleblock}{The Guardian 27.9.2009}
Last week, Iran said it was building a second uranium enrichment plant despite UN demands that it stops its development plans.
\end{exampleblock} \pause

\begin{exampleblock}{Google en-nb}
Sist uke sa \alert{??} at Iran var det å bygge et \alert{sekund} \alert{anriking av uran anlegg} til tross for FNs krav om at den stopper sine utviklingsplaner.
\end{exampleblock} \pause
% doesn't recognize the subject
% doesn't recognize the compound - anriking av uran anlegg

\begin{exampleblock}{Gramtrans en-nb}
Sist uke, sa Iran det holdt på å bygge et annen \alert{uranberigelsesanlæg} tross \alert{UN} krav \alert{som} det stopper sine utviklingsplaner.
\end{exampleblock}
% doesn't translate UN into FN

\begin{exampleblock}{Manuell oversetting}
Forrige uke sa Iran at de var i ferd med å bygge enda en urananrikingsfabrikk til tross for FNs krav om å stoppe utviklingsplanene.
\end{exampleblock}

\end{frame}

\begin{frame}
\begin{exampleblock}{Google nb-en}
Last week, Iran said that they were about to build another \alert{urananrikingsfabrikk} despite UN demands to halt development plans.
\end{exampleblock}

\begin{exampleblock}{Gramtrans nb-en}
\alert{Forrige} week Iran said that they were building still an \alert{urananrichingsfabrikk} in spite of the UN's demand for stopping the development plans.
\end{exampleblock}

\end{frame}

\section{Hva er god oversetting?} 
\begin{frame}\frametitle{Hva er god oversetting?}
\begin{itemize}
\item innhold vs. (kunstnerisk) budskap
\item ordrett vs. fri oversetting 
\item fremmedgjøring eller identifikasjon 
\end{itemize} 
\end{frame}

\begin{frame}
\frametitle{Hva er god oversetting?}
\begin{tabular}{|c|c|}
\hline
\textbf{Språk} & \textbf{Navn} \\
\hline
Italiensk &  Paolino Paperino  \\
\hline
Kroatisk &  Paško Patak  \\
\hline
Kinesisk & Tánglăo Yā \\
\hline
Litauisk & Ančiukas Donaldas \\
\hline
Islandsk & Andrés Önd \\
\hline
Nordsamisk & Vulle Vuojaš \\
\hline
Engelsk & Donald Duck \\
\hline
\end{tabular}
\end{frame}

%\section{Hva er god maskinoversetting?}
%
%\begin{frame}\frametitle{Hva er god maskinoversetting?}
%
%\begin{exampleblock}{Genesis 11,9 (New International Version (NIV))}
%That is why it was called Babel because there the LORD confused the language of the whole world. From 
%there the LORD scattered them over the face of the whole earth.
%\end{exampleblock}
%
%\begin{exampleblock}{manuell oversettelse}
%Derfor kalte de den Babel. For der forvirret Herren all verdens tungemål, og derfra spredte Herren dem ut over hele jorden.
%%-- \url{http://www.bibel.no}
%\end{exampleblock}
%
%\begin{exampleblock}{Google}
%Derfor ble det kalt Babel fordi det Herren forvirret språket i hele verden. Derfra spredte Herren dem over ansiktet til hele jorden.
%\end{exampleblock}
%\end{frame}
 
%\begin{frame}%\frametitle{}
%\begin{itemize}
%\item tekstype: manual, avistekster
%\item assimilsjon - dissiminering
%\end{itemize} 
%\end{frame}
%
%\begin{frame}%\frametitle{}
%\begin{itemize}
%  \item lik antall setninger
%  \item nærmeste oversettelse av ord
%  \item syntaktisk struktur skal være så nært som mulig
%  \item PoS skal være like
%  \item innhold
%\end{itemize}
%\end{frame}
%
%
%\begin{frame}\frametitle{What is machine translation}
%\begin{itemize}
%  \item rule-based vs. statistical
%  \item etc.
%  \end{itemize}
%\end{frame}
%
%
%\begin{frame}\frametitle{The apertium-sme-smj system}
%
%\end{frame}
%
%

%
%\subsection{Lists II}
%\begin{frame}\frametitle{numbered lists}
%\begin{enumerate}
%\item Introduction to  \LaTeX  
%\item Course 2 
%\item Termpapers and presentations with \LaTeX 
%\item Beamer class
%\end{enumerate}
%\end{frame}
%
%\begin{frame}\frametitle{numbered lists with pause}
%\begin{enumerate}
%\item Introduction to  \LaTeX \pause 
%\item Course 2 \pause 
%\item Termpapers and presentations with \LaTeX \pause 
%\item Beamer class
%\end{enumerate}
%\end{frame}
%
%\section{Section no.3} 
%\subsection{Tables}
%\begin{frame}\frametitle{Tables}
%\begin{tabular}{|c|c|c|}
%\hline
%\textbf{Date} & \textbf{Instructor} & \textbf{Title} \\
%\hline
%WS 04/05 & Sascha Frank & First steps with  \LaTeX  \\
%\hline
%SS 05 & Sascha Frank & \LaTeX \ Course serial \\
%\hline
%\end{tabular}
%\end{frame}
%
%
%\begin{frame}\frametitle{Tables with pause}
%\begin{tabular}{c c c}
%A & B & C \\ 
%\pause 
%1 & 2 & 3 \\  
%\pause 
%A & B & C \\ 
%\end{tabular} 
%\end{frame}
%
%
%\section{Section no. 4}
%\subsection{blocs}
%\begin{frame}\frametitle{blocs}
%
%\begin{block}{title of the bloc}
%bloc text
%\end{block}
%
%\begin{exampleblock}{title of the bloc}
%bloc text
%\end{exampleblock}
%
%
%\begin{alertblock}{title of the bloc}
%bloc text
%\end{alertblock}
%\end{frame}
%
%\section{Section no. 5}
%\subsection{split screen}
%
%\begin{frame}\frametitle{splitting screen}
%\begin{columns}
%\begin{column}{5cm}
%\begin{itemize}
%\item Beamer 
%\item Beamer Class 
%\item Beamer Class Latex 
%\end{itemize}
%\end{column}
%\begin{column}{5cm}
%\begin{tabular}{|c|c|}
%\hline
%\textbf{Instructor} & \textbf{Title} \\
%\hline
%Sascha Frank &  \LaTeX \ Course 1 \\
%\hline
%Sascha Frank &  Course serial  \\
%\hline
%\end{tabular}
%\end{column}
%\end{columns}
%\end{frame}
%
%\subsection{Pictures} 
%\begin{frame}\frametitle{pictures in latex beamer class}
%\begin{figure}
%\includegraphics[scale=0.5]{PIC1} 
%\caption{show an example picture}
%\end{figure}
%\end{frame}
%
%\subsection{joining picture and lists} 
%
%\begin{frame}
%\frametitle{pictures and lists in beamer class}
%\begin{columns}
%\begin{column}{5cm}
%\begin{itemize}
%\item<1-> subject 1
%\item<3-> subject 2
%\item<5-> subject 3
%\end{itemize}
%\vspace{3cm} 
%\end{column}
%\begin{column}{5cm}
%\begin{overprint}
%\includegraphics<2>{PIC1}
%\includegraphics<4>{PIC2}
%\includegraphics<6>{PIC3}
%\end{overprint}
%\end{column}
%\end{columns}
%\end{frame}
%
%
%\subsection{pictures which need more space} 
%\begin{frame}[plain]
%\frametitle{plain, or a way to get more space}
%\begin{figure}
%\includegraphics[scale=0.5]{PIC1} 
%\caption{show an example picture}
%\end{figure}
%\end{frame}
%
%
%
%\section{Kontrastiv lingvistikk} 
%\subsection{Emne}
%\begin{frame}\frametitle{unnumbered lists}
%\begin{itemize}
%\item Something about contrastive linguistics
%\item Things we found out about Lule Sámi vs. North Sámi
%\item Orthography - how was the bilingual dictionary made
%\item dat/duot/dat vs....
%\item Negation - tense marker in different elements
%\item SVO vs. SOV
%\item Case Loc vs. Ine/El
%\item Different PoS in some cases gullevaš vs. gullujiddje
%\item Why it can be interesting to work as a linguist -- being a discoverer
%\item Computational systems force you to be accurate -- when you need to be accurate you go into depth of linguistic questions
%\end{itemize} 
%\end{frame}
%
%
%
%
%
%
\end{document}
